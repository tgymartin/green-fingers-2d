\documentclass[12pt,a4paper]{IEEEtran}
\usepackage[latin1]{inputenc}
\usepackage{amsmath}
\usepackage{amsfonts}
\usepackage{amssymb}
\usepackage{graphicx}
\usepackage{float}
\usepackage{gensymb}
\usepackage{hyperref}
\usepackage{cite}
\usepackage[margin=0.6in]{geometry}
\usepackage[justification=centering, font={small,it}]{caption}
\author{Dicson Wijaya (1002289), Wenkie Lau (1002219), \\ Mok Jun Neng (1002219), Charlotte Phang (1002277), \\ Martin Tan (1002173)}
\title{10.012 Introduction to Biology 2D Report 1}
\begin{document}
	
	\maketitle
	
	\section{Objectives} \vspace{-0.5in}
	This report aims to determine the effect of different environmental factors on algal growth. We subjected the different algal flasks to their respective environment over the period of 3 to 4 weeks and periodically measured the growth using a spectrophotometer. \vspace{-0.7in}
	
	\section{Experiment} \vspace{-0.8in}
	\begin{figure}[H]
		\begin{center}
			\includegraphics[width=0.49\textwidth]{algal_samples.png}
			\caption{Algal Samples}
			\label{fig:algalsamples}
		\end{center}
	\end{figure} \vspace{-0.4in}
    Culture Flask A was cultured while placed on a shaker which ensured the constant mixture of the contents of the flask. \\
    
    Culture flask B was cultured in stationary state. \\
    
    Culture flask C was cultured while placed in an environment with strong light intensity. \\
    
    The spectrophotometer results were used to compare the transmittance and absorbency of 3 algal samples grown under different conditions with a control water. \\
    
    The flask with highest algal growth will have the highest transmittance and absorbance.
    
    \section{Experimental Result and Description}
    \begin{figure}[H]
    	\begin{center}
    		\includegraphics[width=0.48\textwidth]{spectro_graph.png}
    		\caption{Graph of spectrophotometer reading against time in days}
    		\label{fig:spectrograph}
    	\end{center}
    \end{figure}
	\begin{table}[H]
		\centering
		\caption{Spectrophotometer readings}
		\label{tab:spectrotable}
		\small
		\begin{tabular}{|c|c|c|c|}
			No. of days & Flask A & Flask B & Flask C \\
			0           & 0.3645  & 0.3945  & 0.3945  \\
			3           & 0.8498  & 0.2786  & 0.1741  \\
			5           & 0.8986  & 0.1634  & 0.2946  \\
			10          & 0.9948  & 0.7528  & 0.7663  \\
			12          & 0.7490  & 0.6481  & 0.4836  \\
			17          & 1.0303  & 0.9359  & 0.8074  \\
			19          & 0.9964  & 0.9217  & 0.7269          
		\end{tabular}
		\normalsize
	\end{table}
	\begin{figure}[H]
		\begin{center}
			\includegraphics[width=0.49\textwidth]{growth_curve.png}
			\caption{Standard Algal Growth Curve}
			\label{growthcurve}
		\end{center}
	\end{figure}
    Comparing Figure 1 and Figure 3: \\
    
    For the first 12 days, the curve representing flask A has a similar shape to that of the standard algal growth curve, with an initial exponential increase in spectrophotometer reading values, followed by a stationary period and finally a period of decline, however, without the initial lag period. However, after the $12^{th}$ day, the number of algal cells decrease for a short period of time, followed by another exponential increase and then a slight decrease recorded during the last reading. \\
    
    Initially, the algal cells number in flask B experienced a decrease, in contrast to the supposed lag time period as predicted by the standard algae growth curve. Subsequently, for both flasks, an exponential
    increase in algal numbers was observed followed by a few cycles of increase and decrease. There was no stationary point observed for the results of both flask B and C. \\
    
    For both flasks A and B, the algae samples experienced positive growth. \\
    
    The increase after day 12 (Flask A) can be attributed to the previous drop in algal cell number. This means that the previous decrease in algal cell numbers have increase the nutrients needed for growth per
    algae cell, causing an increase in algal cells number. The following decrease can be attributed once again to nutrient level per cell, which in this case has decreased due to a higher cell count sharing the same amount of nutrients. It can be expected that cycles of positive growth followed by negative growth will be observed if the experiment is carried out over a longer period of time due to reasons listed above. However, it is more realistic to take the general trend of the curve from day 3 to day 20, which is a gentle increase that can be liken to a plateau representing the stationary portion of the standard algal growth curve. The same explanation can be used to explain the consecutive increase and decrease in algal cells number. \\
    
    Flask C was subjected to artificial lighting at high intensity for all 24 hours of the day.
    
    \section{Analysis of Result}
    \begin{itemize}
    	\item Hypothesis: Under high intensity lighting, the growth rate of algae will increase. Since algae contains chlorophyll~a~and~b, major light harvesting pigments sensitive to light, high intensity lighting allows the algae to photosynthesise continuously and reproduce at a higher average rate. \\
    	\item Observation: Algae sample from flask C changed from a light green colour to a yellow-brown colour. The general spectrophotometer readings for flask C are lower than both flask A and B. \\
    \end{itemize}
    According to our methodology, the higher the spectrophotometer reading, the larger the number of algal cells in the sample. According to this experimental principle, it means that flask C contains the sample with the lowest algal cells count. However, upon deeper analysis, this is an inaccurate method to determine algal growth rate. \\
    
    First of all, the change in the colour of the algal sample in flask C suggest the presence of other impurities, for example, fungi, breeding in the flask. The presence of such organisms affects the spectrophotometer readings which makes the result an inaccurate representation of the number of algal cells in the flask at each reading in comparison to the other flasks. \\
    
    Second of all, the curve of flask C algal growth rate seems to be inconsistent with the standard algae growth curve. However, upon deeper analysis, we observe that flask C's curve has a very similar shape to that of flask B: decline followed by exponential increase and subsequent cycles of decrease and increase in spectrophotometer meeting. From this, it is still possible to argue that algae cells count has increased since spectrophotometer readings indicate that the sample has gotten denser overtime. The discrepancy between our hypothesised outcome and observations can be attributed to the presence of foreign impurities in the flask, thus affecting the transmittance and absorbance of the sample. \\ \vspace{-0.1in}
    
    \section{References}
    \begin{enumerate}
    	\item SYED AZHAR, M. (May 1974). DETERMINATION OF ALGAE GROWTH POTENTIAL IN NATURAL ENVIRONMENT. 1-17.
    	\item Singh, S., \& Singh, P. (2015). Effect of temperature and light on the growth of algae species: A review. Renewable and Sustainable Energy Reviews, 50, 431-444.
    	\item H. (2015, July 05). Enhancing Algal Growth by Stimulation with LED Lighting and Ultrasound.
    \end{enumerate}
    
\end{document}