\documentclass[12pt,a4paper]{IEEEtran}
\usepackage[latin1]{inputenc}
\usepackage{amsmath}
\usepackage{amsfonts}
\usepackage{amssymb}
\usepackage{graphicx}
\usepackage{float}
\usepackage{gensymb}
\usepackage{hyperref}
\usepackage{cite}
\usepackage[margin=0.5in]{geometry}
\usepackage[justification=centering, font={small,it}]{caption}
\author{Dicson Wijaya (1002289), Wenkie Lau (1002219), \\ Mok Jun Neng (1002219), Charlotte Phang (1002277), \\ Martin Tan (1002173)}
\title{10.012 Introduction to Biology 2D \\ Report 1 F06 Team 06}
\begin{document}
	
	\maketitle
	
	\section{Objectives} \vspace{-0.6in}
	This report aims to determine the effect of different environmental factors on algal growth. We subjected the different algal flasks to their respective environment over a period of 3 to 4 weeks and periodically measured the growth using a spectrophotometer. \vspace{-0.8in}
	
	\section{Experiment} \vspace{-0.9in}
	\begin{figure}[H]
		\begin{center}
			\includegraphics[width=0.49\textwidth]{algal_samples.png}
			\caption{Algal Samples}
			\label{fig:algalsamples}
		\end{center}
	\end{figure} \vspace{-0.4in}
    Culture Flask A was cultured while placed on a shaker, giving a constant mixture of contents in the flask. \\ \vspace{-4mm}
    
    Culture Flask B was cultured in stationary state. \\ \vspace{-4mm}
    
    Culture Flask C was cultured while placed in an environment with strong light intensity. \\ \vspace{-2mm}
    
    The spectrophotometer results were used to compare the transmittance and absorbency of 3 algal samples grown under different conditions with a control water. \\ \vspace{-0.4mm}
    
    This experiment works on the basis that the flask with the highest algal growth will have the highest transmittance and absorbance, which will be discussed further in the discussion and analysis section. \vspace{-0.18in}
    
    \section{Experimental Result}
    \begin{figure}[H]
    	\begin{center}
    		\includegraphics[width=0.48\textwidth]{spectro_graph.png}
    		\caption{Graph of spectrophotometer reading against time in days}
    		\label{fig:spectrograph}
    	\end{center}
    \end{figure} \vspace{-0.2in}
	\begin{table}[H]
		\centering
		\caption{Spectrophotometer readings}
		\label{tab:spectrotable}
		\small
		\begin{tabular}{|c|c|c|c|}
			No. of days & Flask A & Flask B & Flask C \\
			0           & 0.3645  & 0.3945  & 0.3945  \\
			3           & 0.8498  & 0.2786  & 0.1741  \\
			5           & 0.8986  & 0.1634  & 0.2946  \\
			10          & 0.9948  & 0.7528  & 0.7663  \\
			12          & 0.7490  & 0.6481  & 0.4836  \\
			17          & 1.0303  & 0.9359  & 0.8074  \\
			19          & 0.9964  & 0.9217  & 0.7269          
		\end{tabular}
		\normalsize
	\end{table}
	\begin{figure}[H]
		\begin{center}
			\includegraphics[width=0.49\textwidth]{growth_curve.png}
			\caption{Standard Algal Growth Curve}
			\label{growthcurve}
		\end{center}
	\end{figure}
    \section{Analysis of Result}
    \subsection{Flask A \& Flask B}
    Comparing Figure 1 and Figure 3: \\
    
    For the first 12 days, the curve representing Flask A has a similar shape to that of the standard algal growth curve, with an initial exponential increase in spectrophotometer reading values, followed by a stationary period and a decline (without the initial lag period). However, after the $12^{th}$ day, the number of algal cells decrease for a short period of time, followed by another exponential increase and a slight decrease recorded during the last reading. \\
    
    Initially, the algal cells number in Flask B experienced a decrease, in contrast to the supposed lag time period as predicted by the standard algae growth curve. Subsequently, for both samples, an exponential
    increase in algal numbers was observed followed by a few cycles of increase and decrease. There was no stationary point observed for the results of both Flask B and Flask C. \\
    
    For both Flask A and Flask B, the algae samples experienced positive growth. \\
    
    The increase after day 12 (Flask A) can be attributed to the previous drop in algal cell number. This means that the previous decrease in algal cell numbers have increased the nutrients needed for growth per
    algae cell, causing an increase in algal cells number. The following decrease can be attributed once again to nutrient level per algae cell, which in this case has decreased due to a higher cell count sharing the same amount of nutrients. It can be hypothesised, though with low probability, cycles of positive growth followed by negative growth will be observed if the experiment is carried out over a longer period of time due to reasons listed above. However, it is more realistic to take the general trend of the curve from day 3 to day 20, which is a gentle increase that can be liken to a plateau representing the stationary portion of the standard algal growth curve. The anomaly observed, a dip in reading value on day 9 can be attributed to experimental error. This is because when negative growth is observed, it is usually associated with a certain requirement that limits cell division or inhibit reproduction, which is often unrelated to environmental factors such as light intensity, temperature, nutrient levels, etc. Thus, we expect a continuous decrease in spectrophotometer readings after the first decline, unlike the observed results in Flask A. The same explanation can be used to explain the consecutive increase and decrease in algal cells number. \\ 
    
    In Flask B, a trend similar to that of Flask A is observed after day 9. In a similar sense, the general trend of increasing spectrophotometer readings, coupled with a dip in recorded value on day 9, can be attributed to either experimental or instrumental error. For day 1 to day 9, the spectrophotometer readings are very much lower than the corresponding values of Flask A. It is possible that this scenario is a result of experimental inaccuracy in which Flask B was not shaken properly with most of the algae resting at the bottom of the flask. This means that when readings are taken, inaccurate samples are being analysed, thus resulting in the discrepancy between the curve of Flask B and the standard algal growth curve.
    
    \subsection{Flask C}
    Flask C was subjected to artificial lighting at high intensity for all 24 hours of the day. \\
    
    \begin{itemize}
    	\item Hypothesis: Under high light intensity, algal growth will increase. Since algae contains chlorophyll~a~and~b, two major light harvesting pigments that are highly sensitive to light, high intensity lighting allows the algae to photosynthesise continuously and reproduce at a higher average rate. \\
    	\item Observation: Algae sample from Flask C changed from a light green colour to a yellowish-brown colour. The general trend of spectrophotometer readings for Flask C is lower than both Flask A and Flask B. \\
    \end{itemize}
    According to our methodology, the higher the spectrophotometer reading, the larger the number of algal cells in the sample. According to this experimental principle, it would imply that Flask C contains the sample with the lowest algal cells count. However, upon deeper analysis, this is an inaccurate method to determine algal growth rate. \\
    
    Firstly, colour change of the algal sample in Flask C suggests the presence of other impurities, for example: fungi, breeding in the flask. The presence of such organisms affects the spectrophotometer readings, making it an inaccurate representation of the number of algal cells in the flask at each data point in comparison to other flasks. \\
    
    Secondly, the plotted curve of algal growth rate for Flask C appears to be inconsistent with the standard algae growth curve. However, upon deeper analysis, we observe that the curve of Flask C is very similar to that of Flask B: a decline followed by exponential increase and subsequent cycles of decrease and increase in spectrophotometer readings. From this, the same argument for Flask B can be applied to Flask C and it is possible to argue that algae cells count has increased since spectrophotometer readings indicate that the sample has gotten denser overtime. The discrepancy between our hypothesised outcome and observations can be attributed to the presence of foreign impurities in the flask, thus affecting the transmittance and absorbance of the sample. \\ \vspace{-0.1in}
    
    \section{References}
    \begin{enumerate}
    	\item Syed Azhar, M. (May 1974). Determination of Algae Growth Potential in Natural Environment. 1-17.
    	\item Singh, S., \& Singh, P. (2015). Effect of temperature and light on the growth of algae species: A review. Renewable and Sustainable Energy Reviews, 50, 431-444.
    	\item H. (2015, July 05). Enhancing Algal Growth by Stimulation with LED Lighting and Ultrasound.
    \end{enumerate}
    
\end{document}