\documentclass[11.5pt,a4paper]{IEEEtran}
\usepackage[latin1]{inputenc}
\usepackage{amsmath}
\usepackage{amsfonts}
\usepackage{amssymb}
\usepackage{graphicx}
\usepackage{float}
\usepackage{gensymb}
\usepackage{hyperref}
\usepackage{cite}
\usepackage[margin=0.25in]{geometry}
\usepackage[justification=centering, font={small,it}]{caption}
\author{Dicson Wijaya (1002289), Wenkie Lau (1002219), \\ Mok Jun Neng (1002219), Charlotte Phang (1002277), \\ Martin Tan (1002173)}
\title{10.012 Introduction to Biology 2D \\ Report 2 F06 Team 06}
\begin{document}
	
	\maketitle
	
	\section{Objectives}
	To use the sulpho-phospho-vanillin method to determine and compare the lipid content in different algae samples. The sulpho-phospho-vanillin assay reacts with lipids to produce colorimetric products of different colour intensity. The products are analysed using a spectrophotometer and the results obtained will be matched against the graph below to determine the lipid content present in a particular sample.
	
	\section{Experimental Result}
	\begin{figure}[H]
		\begin{center}
			\includegraphics[width=0.49\textwidth]{lipid.png}
			\label{fig:lipid}
		\end{center}
	\end{figure}
    \vspace{-8mm} The sulpho-phospho-vanillin method will be administered on the positive controls (known lipid concentrations)to produce colorimetric products with different spectrophotometer readings. The numbers are then used with their respective concentrations to plot a logarithmic graph as shown above. \vspace{-2mm} \\
    
    Reading 1 and 2:
    \begin{figure}[H]
    	\begin{center}
    		\includegraphics[width=0.49\textwidth]{report2_table.png}
    		\caption{A: Positive control, B: Sample 1, C: Sample 2, D: Negative control}
    		\label{fig:table}
    	\end{center}
    \end{figure}
	
    Calculating OD values:

    \begin{table}[H]
    	\centering
    	\label{my-label}
    	\begin{tabular}{|r|r|r|}
    		\hline
    		\multicolumn{1}{|l|}{}      & \multicolumn{1}{l|}{(A1-D1) - (A2-D2)} & \multicolumn{1}{l|}{Concentration} \\ \hline
    		\multicolumn{1}{|c|}{Group} &                                        &                                    \\ \hline
    		2                           & 0.151                                  & 78                                 \\ \hline
    		4                           & 0.645                                  & 156                                \\ \hline
    		6                           & 0.699                                  & 313                                \\ \hline
    		8                           & 1.857                                  & 625                                \\ \hline
    		10                          & 2.49                                   & 1250                               \\ \hline
    	\end{tabular}
    \end{table}
    $Ratio \ of \ B \ to \ C = \frac{(B2-D2)-(B1-D1)}{(C2-D2)-(C1-D1)}$ \\
    
    $\ \ \ \ \ \ \ \ \ \ \ \ \ \ B : C = -1 : 149$                       
    \section{Analysis of Result}
    Our calculated ratio is a negative number and upon analysis, it is highly possible that an experimental error had occur and it would be inaccurate to do an in-depth analysis using our group's results. We will be analysing group 2 results instead. \\
    
    OD value of sample 1: \\
    
    $(B2-D2)-(B1-D1) = 0.299$ \\
    
    From the graph obtained, OD value of 0.299 corresponds to a lipid standard of 112 mg/dL. \\
    
    OD value of sample 2: \\
    
    $(C2-D2)-(C1-D1) = 0.231$ \\
    
    From the graph obtained, OD value of 0.231 corresponds to a lipid standard of 103 mg/dL. \\
    
    Sample one has a higher lipid content compared to sample two. Sample one and sample two are obtained from different strains of algae and this fact can be used to account for the difference. \\
    
    When algal cells are exposed to light, it speeds up the process of reproduction and photosynthesis. This results in the conversion of carbon dioxide into sugar which will then be metabolised by algal cells to form lipids. \\
    
    Different strains of algae require different conditions e.g. different light intensity, carbon dioxide concentration, temperature etc. to achieve optimal growth and photosynthesis rate. It is possible that both algae strains in sample one and two are subjected to the same environment which, in this case, appears to favour the growth of the strain of algae in sample one. This implies that sample one's algal cells have  higher rate of photosynthesis and growth, thus explaining the higher lipid content in sample one compared to sample two. \\
    
    Another possibility could be that one of the samples, in this case, sample one, has been biologically modified possibly for commercial purposes to divert the biosynthetic metabolism of the cells mainly to lipid synthesis. This accounts for the higher lipid content in sample one.
    
    \section{References}
    \begin{enumerate}
    	\item McMahon, A., Lu, H., \& Butovich, I.A. (2013, May). The Spectrophotometric Sulfo-Phospho-Vanillin Assessment of Total Lipids in Human Meibomian Gland Secretions.
    	\item Hannon, M., Gimpel, J., Tran, M., Rasala, B., \& Mayfield, S. (2010, September). Biofuels from algae: challenges and potential.
    \end{enumerate}
\end{document}
