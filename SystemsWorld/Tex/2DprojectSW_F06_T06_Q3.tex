\documentclass[12pt,a4paper]{article}
\usepackage[latin1]{inputenc}
\usepackage{amsmath}
\usepackage{amsfonts}
\usepackage{amssymb}
\usepackage{graphicx}
\usepackage{float}
\usepackage{gensymb}
\usepackage{hyperref}
\usepackage{cite}
\usepackage[margin=0.5in]{geometry}
\usepackage[justification=centering, font={small,it}]{caption}
\author{Dicson Wijaya (1002289), Wenkie Lau (1002219), \\ Mok Jun Neng (1002219), Charlotte Phang (1002277), \\ Martin Tan (1002173)}
\title{Systems World 2D Question 3\\ F06 Team 06}

\begin{document}
	
	\maketitle
	
	\section{Part (a)}
	In order to minimise the number of non-zero coefficients $k_i$'s, we introduce a binary switch $x_i$.  \\
	
	Modifying the chi-square distribution from question 2, equation: \\
	
	$$\sum_{n=0}^{9} \bigg( \sum_{i=0}^{9} \frac{(k_i x_i e(n-i) - v(n))^2}{v(n)} \bigg) + \sum_{i=0}^{9} x_i$$ \\
	
	
	\section{Part (b)}
	We multiplied the tabled chi-square values with the binary switch $x_i$'s. This gives us a 10x1 table of chi-square values from each order controller multiplied by the binary switch. Since we can only display the one value that corresponds to the smallest order controller, we set a constraint: sum of binary $x_i = 1$. Hence, only one chi-square value out of the ten data points, corresponding to the smallest order controller, will appear to be non-zero. Since Excel Solver requires the objective function to be a formula, we simply set it as the sum of chi-square vaues with the binary switch on. \\
	
	From Question 2, a chi-square distribution with 9 degrees of freedom for a lower one-sided test at significance level $\alpha = 0.001$ has a critical value of 1.152. Therefore, a second constraint we set is to have our objective function to be less than or equal to the critical value 1.152. This second constraint sets the ceiling or upper bound for the objective function. \\
	
	The chi-square distribution of calculated v(n) is a convex function. Hence, the smallest order controller is one with a chi-square value just under the ceiling, by the second constraint. Therefore, maximising the objective function gives us the corresponding chi-square value of the smallest order controller: $2^{nd}$ order.
	
	
\end{document}