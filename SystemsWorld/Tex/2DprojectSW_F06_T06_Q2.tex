\documentclass[]{article}
\usepackage{lmodern}
\usepackage{amssymb,amsmath}
\usepackage{ifxetex,ifluatex}
\usepackage{fixltx2e} % provides \textsubscript
\ifnum 0\ifxetex 1\fi\ifluatex 1\fi=0 % if pdftex
  \usepackage[T1]{fontenc}
  \usepackage[utf8]{inputenc}
\else % if luatex or xelatex
  \ifxetex
    \usepackage{mathspec}
  \else
    \usepackage{fontspec}
  \fi
  \defaultfontfeatures{Ligatures=TeX,Scale=MatchLowercase}
\fi
% use upquote if available, for straight quotes in verbatim environments
\IfFileExists{upquote.sty}{\usepackage{upquote}}{}
% use microtype if available
\IfFileExists{microtype.sty}{%
\usepackage[]{microtype}
\UseMicrotypeSet[protrusion]{basicmath} % disable protrusion for tt fonts
}{}
\PassOptionsToPackage{hyphens}{url} % url is loaded by hyperref
\usepackage[unicode=true]{hyperref}
\hypersetup{
            pdfborder={0 0 0},
            breaklinks=true}
\urlstyle{same}  % don't use monospace font for urls
\usepackage{longtable,booktabs}
% Fix footnotes in tables (requires footnote package)
\IfFileExists{footnote.sty}{\usepackage{footnote}\makesavenoteenv{long table}}{}
\IfFileExists{parskip.sty}{%
\usepackage{parskip}
}{% else
\setlength{\parindent}{0pt}
\setlength{\parskip}{6pt plus 2pt minus 1pt}
}
\setlength{\emergencystretch}{3em}  % prevent overfull lines
\providecommand{\tightlist}{%
  \setlength{\itemsep}{0pt}\setlength{\parskip}{0pt}}
\setcounter{secnumdepth}{0}
% Redefines (sub)paragraphs to behave more like sections
\ifx\paragraph\undefined\else
\let\oldparagraph\paragraph
\renewcommand{\paragraph}[1]{\oldparagraph{#1}\mbox{}}
\fi
\ifx\subparagraph\undefined\else
\let\oldsubparagraph\subparagraph
\renewcommand{\subparagraph}[1]{\oldsubparagraph{#1}\mbox{}}
\fi

% set default figure placement to htbp
\makeatletter
\def\fps@figure{htbp}
\makeatother


\date{}

\begin{document}

\emph{\textbf{2D Project}}

\emph{\textbf{Systems World}}

\textbf{F06 Team 06}

Dicson Candra Wijaya 1002289

Lau Weng Kei Wenkie 1002219

Phang Li Wen Charlotte 1002277

Mok Jun Neng 1002286

Tan Guan Yuan Martin 1002173

\emph{\textbf{Question 2.}}

\emph{Our group used Excel's matrix multiplication (=MMULT) of the
entries of error and the k coefficients to get the linear combination of
calculated v(n) for every order controller.}

\emph{We decided to use minimum chi-square in assessing the goodness of
fit between the set of calculated values of v(n) and actual values of
v(n) given in the question.}

\emph{Pearson's Chi-Square Test: }

\(X^{2} = \sum_{i = 1}^{n}\frac{\left( O_{i} - E_{i} \right)^{2}}{E_{i}}\)

\emph{where}

\emph{X\textsuperscript{2} = Pearson's cumulative test statistic}

\emph{O\textsubscript{i} = calculated v(n)}

\emph{E\textsubscript{i} = actual v(n) given}

\emph{n = number of cells in the table}

\emph{A lower X\textsuperscript{2} value indicates that the calculated
v(n) is closer to the actual v(n) given. Therefore, our objective is to
minimize the chi-square value (X\textsuperscript{2}) for each order
controller.}

\emph{Using Excel Solver, we set the chi-square value
(X\textsuperscript{2}) as the objective function to a minimum and set
the unknown k coefficients (k\textsubscript{0},
k\textsubscript{1},\ldots{},k\textsubscript{9}) as the changing
variable.}

\emph{Result:}

\begin{quote}
{{[}CHART{]}}
\end{quote}

\begin{longtable}[]{@{}ll@{}}
\toprule
\emph{Order} & \emph{X\textsuperscript{2}}\tabularnewline
\midrule
\endhead
0 & 20.26071659\tabularnewline
1 & 3.089836451\tabularnewline
2 & 0.006754156\tabularnewline
3 & 0.006716346\tabularnewline
4 & 0.006493217\tabularnewline
5 & 0.005102069\tabularnewline
6 & 0.003122692\tabularnewline
7 & 0.002727727\tabularnewline
8 & 0.000374947\tabularnewline
9 & 3.57577E-10\tabularnewline
\bottomrule
\end{longtable}

\emph{The next step is to analyze the result obtained. Since 10 data set
is given, the degree of freedom (N-1) is 9. A chi-square distribution
with 9 degrees of freedom for a lower one-sided test at significance
level α = 0.001 (extreme) has a critical value of 1.152.}

\emph{From the table of values, we can see that the minimized chi-square
values (X\textsuperscript{2}) of 2\textsuperscript{nd} order onwards do
not exceed this critical value and hence satisfy the condition to be
considered ``correct''. However, it is also apparent, from the table of
values and the graph, that X\textsuperscript{2} already approaches 0 at
2\textsuperscript{nd}order and rate of change of X\textsuperscript{2}
from 2\textsuperscript{nd} order onwards to the 9\textsuperscript{th}
order is negligible. }

\emph{Therefore, we conclude that the team from last year's 2D project
used controller of 2\textsuperscript{nd} order.}

\emph{This approach is still reasonable even if we have more data
points. In fact, minimized chi-square test becomes more relevant with
increasing data points. The plot of X\textsuperscript{2} vs order would
give data points forming the full similarly converging curve. With more
data points, there is increased precision in assessing the data point
which chi-square value (X\textsuperscript{2}) first converges to
approximately 0. Therefore, the chi-square test is a reliable tool for
assessing the goodness of fit between observed and expected values.}

\emph{The k coefficients in the 2\textsuperscript{nd} order controller
is:}

\begin{longtable}[]{@{}ll@{}}
\toprule
\emph{k\textsubscript{0}} & 0.107782668\tabularnewline
\midrule
\endhead
\emph{k\textsubscript{1}} & 0.546854887\tabularnewline
\emph{k\textsubscript{2}} & 0.769143476\tabularnewline
\bottomrule
\end{longtable}

\emph{We have observed that the coefficients of k\textsubscript{0} and
k\textsubscript{1} are the same for 2\textsuperscript{nd} and
3\textsuperscript{rd} order controller, with a different coefficient of
k\textsubscript{2} because the 3\textsuperscript{rd} order controller
has an additional k\textsubscript{3} coefficient. The unchanged values
of k\textsubscript{0} and k\textsubscript{1} suggest that the
2\textsuperscript{nd} order controller is indeed sufficient and these k
coefficients are ``best'' or true.}

\emph{Therefore, we conclude that the sufficient 2\textsuperscript{nd}
order controller was used and the corresponding k coefficients are
correct.}

\emph{\emph{Bibliography}}

Appendix C Critical Values for the Chi-Squared Distribution.
(2015).~\emph{Beyond Basic Statistics,}167-167.

Moore, D. S. (1976).~\emph{Chi-Square Tests}~(No. Mimeograph). PURDUE
UNIV LAFAYETTE IND DEPT OF STATISTICS.

\end{document}
